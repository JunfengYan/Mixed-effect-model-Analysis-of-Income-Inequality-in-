% Options for packages loaded elsewhere
\PassOptionsToPackage{unicode}{hyperref}
\PassOptionsToPackage{hyphens}{url}
%
\documentclass[
]{article}
\usepackage{lmodern}
\usepackage{amssymb,amsmath}
\usepackage{ifxetex,ifluatex}
\ifnum 0\ifxetex 1\fi\ifluatex 1\fi=0 % if pdftex
  \usepackage[T1]{fontenc}
  \usepackage[utf8]{inputenc}
  \usepackage{textcomp} % provide euro and other symbols
\else % if luatex or xetex
  \usepackage{unicode-math}
  \defaultfontfeatures{Scale=MatchLowercase}
  \defaultfontfeatures[\rmfamily]{Ligatures=TeX,Scale=1}
\fi
% Use upquote if available, for straight quotes in verbatim environments
\IfFileExists{upquote.sty}{\usepackage{upquote}}{}
\IfFileExists{microtype.sty}{% use microtype if available
  \usepackage[]{microtype}
  \UseMicrotypeSet[protrusion]{basicmath} % disable protrusion for tt fonts
}{}
\makeatletter
\@ifundefined{KOMAClassName}{% if non-KOMA class
  \IfFileExists{parskip.sty}{%
    \usepackage{parskip}
  }{% else
    \setlength{\parindent}{0pt}
    \setlength{\parskip}{6pt plus 2pt minus 1pt}}
}{% if KOMA class
  \KOMAoptions{parskip=half}}
\makeatother
\usepackage{xcolor}
\IfFileExists{xurl.sty}{\usepackage{xurl}}{} % add URL line breaks if available
\IfFileExists{bookmark.sty}{\usepackage{bookmark}}{\usepackage{hyperref}}
\hypersetup{
  pdftitle={Explaining income inequality in Europe using longitudinal data},
  pdfauthor={Andra-Ecaterina Boca, Junfeng Yan, Thien An Pham},
  hidelinks,
  pdfcreator={LaTeX via pandoc}}
\urlstyle{same} % disable monospaced font for URLs
\usepackage[margin=1in]{geometry}
\usepackage{graphicx,grffile}
\makeatletter
\def\maxwidth{\ifdim\Gin@nat@width>\linewidth\linewidth\else\Gin@nat@width\fi}
\def\maxheight{\ifdim\Gin@nat@height>\textheight\textheight\else\Gin@nat@height\fi}
\makeatother
% Scale images if necessary, so that they will not overflow the page
% margins by default, and it is still possible to overwrite the defaults
% using explicit options in \includegraphics[width, height, ...]{}
\setkeys{Gin}{width=\maxwidth,height=\maxheight,keepaspectratio}
% Set default figure placement to htbp
\makeatletter
\def\fps@figure{htbp}
\makeatother
\setlength{\emergencystretch}{3em} % prevent overfull lines
\providecommand{\tightlist}{%
  \setlength{\itemsep}{0pt}\setlength{\parskip}{0pt}}
\setcounter{secnumdepth}{-\maxdimen} % remove section numbering

\title{Explaining income inequality in Europe using longitudinal data}
\author{Andra-Ecaterina Boca, Junfeng Yan, Thien An Pham}
\date{March 15, 2020}

\begin{document}
\maketitle
\begin{abstract}
Income inequality has long been the domain of macroeconomics and
standard labor economics. Literature postulates that GDP growth, the
openness of an economy, and its labor market structure as potential
drivers of country-level inequality. We test this using a Mixed Effects
approach in 36 select European countries with data collected by the
World Bank. We find that although the strongest predictors of income
inequality are solely a country's GDP per capita and employment in
industry, a model accounting for all the previously-mentioned factors
can be argued to be preferable over a simpler one.
\end{abstract}

\hypertarget{introduction}{%
\subsection{Introduction}\label{introduction}}

\hypertarget{literature-review}{%
\subparagraph{Literature review}\label{literature-review}}

With more countries experiencing unprecedented rates of growth, wealth
inequality has become a more and more pressing issue. Wealth disparities
can appear due to market distortions such as corruption or
non-productive economic activity like rent-seeking, which we describe
dedicating resources above what is necessary for keeping an asset alive
(for example, extension of copyright law). Income inequality on the
other hand tends to be a more insidious topic of interest. Within income
inequality, we distinguish between earnings from assets such as
inherited wealth and wage compensation. Some findings suggest that
wealth inequality is disproportionately influenced by market-driven
factors, through the relative gains made by the top of the income
distribution rather than by diverging levels of compensation -- as such,
jobs such as superstar athletes or Wall Street employees make
significantly more money due to market forces that set their fees or
performance at historically unprecedented prices (Kaplan 2013).

On the other hand, there still remains a large proportion of variability
in earnings that is not necessarily explained by the rise of `superstar'
professionals. In the United States, the minimum wage has been declining
in real terms for decades (Autor 2016) and the rising competition in the
developing world and advanced technology have created an incentive for
developed countries to outsource and automate lower-skilled labor. To
that end, focusing on the incomes of the top 1\% relative to the rest of
the population clouds our understanding of emerging compensating
differentials within that `99\%' demographic.

The European Union and other related European countries represents an
important area to study in the domain of wage inequality. It is first
and foremost a diverse space in terms of population demographics, labor
laws, and even culture around work. On the other hand, it mostly acts as
a unitary space of population and capital movement that correlate any
changes we might see in wage inequality across time and across
countries. On average, wage inequality is expectedly lower in Europe
relative to the US given better labor laws coverage and more widespread
collective bargaining (DiPrete 2008). Using a microeconomic
household-level lens, Rodríguez‐Pose (2009) showed that European income
per capita relationship with income inequality is positive while
educational outcomes are not significantly correlated with inequality in
the long-run. In terms of collective bargaining, Dell'Aringa (2005)
found that multi-bargaining systems (more centralized worker bargaining)
might have an effect on wage dispersion in European context.

At a macroeconomic level, Budría (2005) found that returns to tertiary
education play a significant role in transitioning from the bottom
quintiles of the income distribution to the upper quintiles in 8 select
European countries. The rising inequality does not come from lower
employment in lower-skilled occupations: in fact, a study for the United
Kingdom shows employment has increased in top-paid jobs (professional
and management) as well as in the lowest-paid jobs (such as cleaning and
personal care) according to Goos (2007). To that end, it is worth
empirically examining the structure of employment in a country as a
contributing factor to income inequality.

\hypertarget{theoretical-framework}{%
\subparagraph{Theoretical framework}\label{theoretical-framework}}

In order to study income inequality at an aggregated level such as
country data, we need to take two theoretical views on the problem.
Firstly, we need to account for the variation in wage dispersion as
suggested by this strand of labor economics and secondly, we need to be
able to model inequalities arising from changing macroeconomic
fundamentals, such as productivity and trade at a country level, which
are associated with economic development.

Traditionally, what has been used to model wealth inequality in
developing countries is the Kuznets curve formalized by Simon Kuznets in
the 1950's. This formula posits that countries tend to become more
unequal as they experience increased productivity and growth (that is,
as they experience development), and then see a decrease in inequality.
This is theorized to appear due to the level of opening of an economy
(pictured by exports, imports and foreign direct investment) that
indicate that firms in developed economies respond to import competition
from low-wage countries by moving non-skill-intensive activities abroad.
The Kuznets curve is expected to showcase a U-shape where inequality
rises as the transition from agricultural to industrial jobs and from
rural to urban happens, and decreases back to its original starting
point as income per capita increases to the levels of what are
considered developed countries.

From a labor-focused theoretical standpoint, wage inequality is to be
expected due to different compensation for differing levels of skill.
Since the 1990's, a significant strand of literature has tried to
explain changing compensation levels by the skill-based technological
change (SBTC) hypothesis -- that is, the tendency of developed economies
to shift demand towards more skilled and educated workers (for a
comprehensive literature survey see Katz (1999)). This explanation has
been in competition with previous supply-side considerations: factors
such as collective bargaining, unionization, wage differentials across
industries that were mainly shifted by workers (the labor supply),
rather than firms and consumers.

Based on previous literature, it remains crucial for the study of income
inequality to model the differences across countries in their
populations' labor earning potential. This paper takes a Mixed Effects
approach to analyse the drivers of inequality in 36 European countries.
We use World Bank longitudinal data on select macroeconomic variables
between the years 2005 and 2016. Our outcome variable of interest is the
Gini country-level coefficient tracked by Eurostat for the European
Statistics on Income and Living Condition (EU-SILC) survey which was
then merged to relevant World Bank indicators.

Specifically, this analysis can contribute to existing literature by
providing a unique country-by-country lens to income inequality. While
it might not allow us to draw any broad conclusions about the dynamics
of this economic indicator, the analysis using a Mixed Effects approach
can give us an insight into across-country and across-time variation in
the Gini inequality coefficient. Based on previous literature that
hypothesizes that GDP growth, the openness of an economy, and its labor
market structure as potential drivers of country-level inequality, we
test the null hypothesis that these variables have no effect on the Gini
inequality index.

We discuss data and statistical methodology in \textbf{Section II},
results in \textbf{Section III} and give a discussion about conclusions
and limitations of our model in \textbf{Section IV}. We test our
hypothesis using a Mixed Effects approach in 36 select European
countries with data collected by the World Bank. We find that although
the strongest predictor of income inequality is employment in industry,
a model accounting for all the previously-mentioned factors is
statistically preferred over a simpler one.

\hypertarget{methods}{%
\subsection{Methods}\label{methods}}

\hypertarget{data-description}{%
\subparagraph{Data description}\label{data-description}}

The macroeconomic indicator data for the analysis comes from the World
Bank DataBank publicly available data. It tracks macroeconomic variables
for 36 selected countries across 11 years (2005-2016). The countries we
are interested in are both part of the European Union (EU) and outside
at different points in time (for example, Bulgaria and Romania accede to
the EU in 2007). These countries are: Albania, Austria, Belgium,
Bulgaria, Croatia, Cyprus, Czech Republic, Denmark, Estonia, Finland,
France, Germany, Greece, Hungary, Iceland, Ireland, Italy, Latvia,
Lithuania, Luxembourg, Malta, Montenegro, Netherlands, North Macedonia,
Norway, Poland, Portugal, Romania, Serbia, Slovak Republic, Slovenia,
Spain, Sweden, Switzerland, Turkey, United Kingdom.

The outcome variable is the \textbf{Gini inequality coefficient} tracked
by Eurostat. The Gini coefficient is a metric of statistical dispersion
for wealth. In percentage terms, a Gini coefficient of 0 represents a
perfectly equal country showcasing perfect homogeneity in income across
a distribution, while a Gini coefficient of 100 represents complete
inequality. Eurostat considers income everything encompassing wages,
self-employment earnings, private income from investment and property,
transfers between households (such as remittances or gifts) and social
transfers (pensions, benefits and others). For a more in-depth
description of all the other variables, please consult the Appendix.

\hypertarget{descriptive-statistics}{%
\subparagraph{Descriptive statistics}\label{descriptive-statistics}}

To be able to better understand the data, we include some visual
descriptions of the development across time of select variables and
their relationships with our outcome variable (the Gini coefficient).

\includegraphics{FinalReport_files/figure-latex/unnamed-chunk-4-1.pdf}

\includegraphics{FinalReport_files/figure-latex/unnamed-chunk-5-1.pdf}

In \textbf{Figure 1} and \textbf{Figure 2}, we plot the change in
inequality measured by our outcome variable of interest, the Gini
coefficient. It is notable that countries that are in transition from
developing to developed (primarily Eastern European countries) tend to
showcase larger inequality and variability, as opposed to more developed
Western European countries in \textbf{Figure 2}. This is in line with
the Kuznets economic hypothesis outlined in the \textbf{Section 1}.
Typically, we see increased inequality for countries in transition.

In terms of the explanatory variables we consider based on the theory,
we note that most countries have similar trajectories in GDP per capita
growth, unemployment or employment in Industry for example. Most of the
other macroeconomic health variables in the dataset follow the same
parallel pattern of stable growth or decrease. An example for GDP per
capita in \textbf{Figure 3} showcases such trajectory.

\includegraphics{FinalReport_files/figure-latex/unnamed-chunk-6-1.pdf}

Two particular variables require a more careful analysis. In
\textbf{Figure 5} and \textbf{Figure 6}, we can see the trajectories of
foreign direct investment (FDI) and trade across time might be skewed by
outliers if specified as fixed in our model. In the case of both FDI and
trade, Luxembourg is quite starkly different to all the other countries.
Luxembourg is a small country with a small production level on its own,
and as such the large amount of trade and FDI to the country represent
outliers as a ratio of a small GDP. Malta and Cyprus are similar cases
in terms of FDI and trade respectively. For our analysis, this means the
relatively larger starting points of these countries show a correlation
to their growth trajectory.

\includegraphics{FinalReport_files/figure-latex/unnamed-chunk-7-1.pdf}

\includegraphics{FinalReport_files/figure-latex/unnamed-chunk-8-1.pdf}

\includegraphics{FinalReport_files/figure-latex/unnamed-chunk-9-1.pdf}

\textbf{Figure 7} showcases the relationship between different GDP per
capita and inequality value pairs for each country. We note a distinctly
decreasing pattern. As GDP increases, the Gini coefficient decreases or,
in other words, inequality decreases. We expect to see poorer countries
have a larger inequality index.

\includegraphics{FinalReport_files/figure-latex/unnamed-chunk-10-1.pdf}

Similarly, we expect that countries with a higher level of population
involved in economic activities away from agriculture and industry will
perform better in terms of productivity. We note the same decreasing
pattern between engagement in services and inequality as in
\textbf{Figure 8}.

In selecting what variables would best represent a country's employment
and labor structure, we find that net investment as a percentage of GDP,
tax revenue, education expenditure as a percentage of total government
expenditure do not vary significantly with inequality. These plots are
available in \textbf{Section 2} of the Appendix. We then assume they are
artifacts of each country and also stay quite constant over time. In
terms of demographics, we find that net migration has a weak
relationship with inequality, but we choose not to focus on this
variable in our final model.

\hypertarget{statistical-methodology}{%
\subparagraph{Statistical methodology}\label{statistical-methodology}}

As per our literature review and theoretical framework, GDP growth, the
openness of an economy, and its labor market structure are the three
most frequently identified drivers of wage inequality in an economy. As
such, we use the log of GDP per capita to represent economic growth,
trade value and FDI to represent openness of economy , percentage of
employment in industry, percentage of employment in service and in
industry, and percentage of population in urban areas to account for
different demographic structure among the observed countries.
Additionally, we created a dummy variable for year 2008 to account for
the great recession. Our baseline model of wage inequality,represented
using the gini coefficient, is written as such:

\[ gini = \beta_0 + \beta_1*{lggdp per capita} + \beta_2*{trade} + \beta_3{FDI} + \beta_4{industryemployment} + \beta_5{employmentservice}+\beta_6{urbanpopulation} + \beta_7*year2008\]

In other words, this is a naive OLS model. However,given the
longitudinal nature of our data,which has regular repeated observations
of different countries, this method will clearly give biased results. We
then turn to marginal models with generalized estimating equations (GEE)
(Liang and Zeger 1986) and mixed-effect models (Laird, Ware, and others
1982) to tackle this regression questions. In short, GEE estimate the
sample population average coefficient for a given set of parameters and
produces robust standard errors for each parameter, due to the fact that
it specifies a working correlation structure for each observed
individual. On the other hand, mixed effect model allows for
heterogeneity within the estimated intercept as well as some regression
coefficients for the sample population (random effect),in addition to
estimating the population average for a given set of parameters (fixed
effect).

For our specific research question, the mixed effect model is preferred.
As informed by \textbf{Figure 1}, the observed countries clearly started
at different level of Gini coefficient at the start of coefficient
period. Moreover, it is apparent in \textbf{Figure 5} that different
countries exhibit different trade volume and FDI trajectory within our
observation period. It is therefore necessary to introduce
country-specific random intercept, denoted as \(b_i\), as well as
country-variable-specific random coefficients,denoted as \(\beta_{ij}\)
to account inter-country differences in terms of starting level and
variable coefficients. Both random effects are assumed to follow a
probability distribution, such as \(N ~ (0,\sigma_i)\). In addition, for
a sufficiently complex phenonemon such as inter-coutnry income
inequality, it is to some degree inevitable that there are unobserved
factors or unobserved differences between country that are significant
to income inequality. Country-level random intercept and coefficient can
to some degree accounted for this potential omitted variable bias. Our
tentative models are therefore:

\[ model 1: gini = \beta_0 + \beta_1*{lggdp per capita} + \beta_2*{trade} + \beta_3{FDI} + \beta_4{industryemployment} +\]
\[+ \beta_5{employmentservice}+\beta_6{urbanpopulation} + \beta_7*year2008 + b_i\]

and

\[ model 2: gini = \beta_0 + \beta_1*{lggdp per capita} + \beta_2*{trade} + \beta_3{FDI} + \beta_4{industryemployment} + \beta_5{employmentservice} + \]
\[+ \beta_6{urbanpopulation}+ \beta_7*year2008 + b_i + b_{i,trade}\]

After using maximum likelihood to estimate our model, we however find
that the random effect and random intercept for model 2 has a
correlation of 0.92, meaning they are perfectly negatively correlated
with each other. Although we are not sure of the exact cause behind this
estimates, it is discretionary to avoid estimates that are close to the
boundary of possible values and we therefore discarded model 2. Out of
discretion, we also fit the model with random slopes for every other
variable. It appears that every random effect that we estimated are met
with similar problems.

With only model 1 left, we now concern ourselves with this problem: is
the full specification necessary or is a simpler version of the model
preferable. To shrink down the number of specified variables, we first
conduct a single variable hypothesis test with null hypothesis
\(H0 : \beta = 0\). We can calculate a z-statistic,
\(z = \frac{\hat{SE}}{\hat{\beta}}\). This z-statistic is simply given
by our model output. We assume that significant variables will have a
z-score that is larger than 1.96, which translates to a p-value smaller
than 0.05 if we were to assume the sample distribution to be normal. We
however, also retain trade volume since its theoretical importance and
its z-statistic are close to our threshold value. With this test, the
strongest three predictors are the log of gdp per capital, employment in
industry and trade volume.The simpler model is therefore:

\[
model 3:  gini = \beta_0 + \beta_1*{lggdp per capita} +\beta_2{trade}+ \beta_4{industryemployment}+b_i
\]

Running the model again with the simpler specification also produces
lower AIC and BIC than the full specification. It is important to note
that now trade has a significant z-statistic.

However, we did not simply conclude that the simpler specification is
preferable over the full specification. It is important to stress again
the focal point of this paper: we are attempting to explain main driver
of income inequality instead of building a predictive model of income
inequality. As such, it is undesirable to discard variables that theory
suggests to be significant simply due to insufficiently large
z-statistics. To test a more involved hypothesis,
\(H0 : \mathbf L \beta = 0\), that whether multiple slopes are 0, we
calculate a W-statistic as follows:

\[
\mathbf W^2 = (\mathbf L \hat\beta)^T(\mathbf L \hat{Cov(\hat\beta)}\mathbf L^{\mathbf T})^{-1}(\mathbf L \hat\beta)
\]

We then used this test to see whether the slopes of variables that did
not pass the individual hypothesis test, namely trade volume, foreign
direct investment, service employment, urban population, and year 2008
are in fact 0. With this test, we obtain a p-value of 0.01754489, which
means that we have significant evidence to reject the null hypothesis
that these slopes are 0. With our theoretical consideration and this
test result, we conclude our final model to be the first model that
includes all the variables.

\hypertarget{model-diagnostics}{%
\subparagraph{Model Diagnostics}\label{model-diagnostics}}

Since mixed-effect model assume errors, denoted \(\epsilon_i\), as well
as the random intercept to be normally distributed and to have a mean 0,
we lastly graph a residual plots and QQ plots to see if this assumption
is being violated.

\includegraphics{FinalReport_files/figure-latex/unnamed-chunk-11-1.pdf}
\includegraphics{FinalReport_files/figure-latex/unnamed-chunk-11-2.pdf}

Residuals plots shows that \(\epsilon_1\) have mostly no pattern and is
balanced 0. The QQ plot appears a bit more concerning as the random
intercepts appear to stray from the normal line at both ends.

\hypertarget{results}{%
\subsection{Results}\label{results}}

As discussed above, we found that the best model for our data is the
random intercept and slope one where we determine wage inequality by
measuring log of GDP per capita, a country's trade, a country's foreign
direct investment, unit percentage of employment in a country's
industries, unit percentage of employment in a country's industries,
unit percentage of employment in a country's services, percentage of
urban population in a country, and whether or not a country experienced
an economic crisis in 2008. We can see that the variance is pretty high
of 10.645, suggesting wage inequality varies a lot by countries.

\begin{verbatim}
## Warning: package 'sjPlot' was built under R version 3.6.3
\end{verbatim}

\begin{verbatim}
## Warning: package 'sjmisc' was built under R version 3.6.3
\end{verbatim}

~

Gini coefficient

Predictors

Estimates

CI

p

(Intercept)

59.90

48.53~--~71.26

\textless0.001

Log GDP per capita

-1.60

-3.06~--~-0.14

0.031

Percent Urban Population

-0.01

-0.02~--~0.00

0.060

Foreign Direct Investment

-0.00

-0.01~--~0.00

0.612

Employment Industry

-0.20

-0.34~--~-0.05

0.007

Trade

-0.06

-0.18~--~0.06

0.335

Employment Services

-2.38

-11.49~--~6.72

0.608

Year 2008

0.08

-0.37~--~0.53

0.731

Random Effects

σ2

1.43

τ00 id

10.64

ICC

0.88

N id

36

Observations

365

Marginal R2 / Conditional R2

0.212 / 0.906

By looking at the p values, log of GDP per capita of a country and unit
percentage of employment in a country's industries are two statistically
significant predictors that have p values of less than 0.05 - 0.031 and
0.07 respectively. The fixed effects also tell us that these are the
only two predictors with more than two standard deviations away.

\begin{itemize}
\item
  \textbf{logGDPpercapita}: if a country's log of GDP per capita
  increases by 1 unit, their wage inequality is expected to decrease by
  1.60\%.
\item
  \textbf{trade}: if a country trades 1\% more, their wage inequality is
  expected to decrease by 0.01\%.
\item
  \textbf{fdi}: if a country invests in another country 1\% more, their
  wage inequality is expected to remain unchanged.
\item
  \textbf{emplInd}: if a country has 1\% higher employment in their
  industries, their wage inequality is expected to decrease by 0.20\%.
\item
  \textbf{emplServ}: if a country has 1\% higher employment in their
  services, their wage inequality is expected to decrease by 0.06\%.
\item
  \textbf{percUrbanPop}: if a country's population in urban areas
  increases by 1\%, their wage inequality is expected to decrease by
  2.38\%.
\item
  \textbf{year2008}: if the year of the observation is 2008
  (\texttt{year2008} = 1), the country's wage inequality is expected to
  increase by 0.08\%.
\end{itemize}

The output also tells us the correlation of fixed effects, or the
expected correlation of the regression coefficients. For example, if the
coefficient for the log of GDP per capita increases by 1 unit, it is
likely that the coefficient for trade decreases by 0.076\% and vice
versa. Or if the coefficient for trade increases by 1\%, it is likely
that the coefficient for percentage of employment in a country's
industries increases by 0.12\%. This may seem to represent
multicollinearity but not necessarily. It tells you that should you test
the model again and the coefficient changes, other correlated
coefficients may change as well in a positive or negative direction.

\hypertarget{conclusions}{%
\subsection{Conclusions}\label{conclusions}}

Using World Bank and Eurostat data between 2005 and 2011, we test
whether a country's GDP growth, trade openness, and its labor market
structure have a significant effect on its economic inequality using a
Mixed Effects approach. We find that although the strongest predictors
of income inequality are a country's GDP per capita and employment in
industry, a model accounting for all the previously-mentioned factors
can be argued to be preferable over a simpler one in referral to
economic theory.

\hypertarget{limitations}{%
\subparagraph{Limitations}\label{limitations}}

First, it is important to keep in mind the limited scope of our project.
We use a theoretical model that seeks to prove a very widely-reaching
hypothesis. The Kuznets curve variables are hypothesized for the life
cycle of a country from industrialization to its actual observed growth
which can last decades -- we are, on the other hand, looking at a mere
11 years of country-level data. While we do see some significant effects
across countries, the development path of a single country in this
limited time frame might not tell us so much. Similarly, there remains a
lot of omitted variable bias that we should correct for, especially in
terms of demographics. Controlling for an aging population, for example,
might result in a different statistical significance for our variables
of interest.

Secondly, as shown in the descriptive statistics in Section 3, there are
outlier countries such as Luxembourg and Malta that might drive us to
different analysis results when we add trade and FDI into our model.
More data work with and without the outliers might lead to more robust
results.

Thirdly, our model diagnostics, specifically the QQ plot, show that the
random intercept at smaller or larger values may not follow a normal
distribution. Although this might undermine the validity of our model,
with more rigorous statistical method or tools we can account for such
limitation.

\hypertarget{acknowledgements}{%
\subsection{Acknowledgements}\label{acknowledgements}}

We want to thank everyone that made this paper possible: first and
foremost, our STAT 494 Correlated Data professor Brianna Heggeseth for
her patient guidance and help every time we were stuck, as well as the
class preceptors, Yuren (Rock) Pang, Raven McKnight and Kieran Liming,
for their continuous support.

\hypertarget{appendix}{%
\subsection{Appendix}\label{appendix}}

\hypertarget{section-1-variable-description}{%
\subparagraph{Section 1: Variable
description}\label{section-1-variable-description}}

We track the following variables in our compiled World Bank dataset:

\begin{itemize}
\item
  Population: population of country in absolute terms at given year
  Urban population: population of a country living in a city in absolute
  terms at given year
\item
  Percent urban population: ratio of urban population to total
  population (\%)
\item
  Imports: all the goods imported from the rest of the world to specific
  country reported as percentage of GDP (\%)
\item
  Exports: all the goods exported to the rest of the world to specific
  country reported as percentage of GDP (\%)
\item
  Trade: the sum of exports and imports of goods and services as a share
  of gross domestic product; reported as a percentage of GDP (\%)
\item
  Foreign direct investment: net inflows of investment to a country,
  reported as a percentage of GDP (\%)
\item
  GDP (Gross Domestic Product): sum of value added by all the producers,
  sometimes referred to as productivity or growth, reported in constant
  2010 USD
\item
  GDP per capita (Gross Domestic Product per capita): gross domestic
  product divided by midyear population, reported in constant 2010 USD
\item
  Net investment in non-financial assets: investment in fixed assets,
  inventories, valuables, and non-produced assets, reported as a
  percentage of GDP (\%)
\item
  Education expenditure: government expenditure on education, reported
  as a percentage of GDP (\%)
\item
  Unemployment: total population out of labor force that is currently
  unemployed, reported as a percentage of labor force (\%)
\item
  Tax revenue: compulsory transfers to the central government for public
  expenditure purposes, reported as a percentage of GDP (\%)
\item
  Employment in agriculture/industry/services: should sum up to 100\%,
  reported as a percentage of total employment (\%)
\item
  Net migration: the number of immigrants minus the number of emigrants,
  in absolute terms.
\item
  Wage inequality: the 80/20 wage inequality ratio calculated by
  Eurostat.
\end{itemize}

\hypertarget{section-2-visual-relationships-of-variables-and-gini-coefficient}{%
\subparagraph{Section 2: Visual relationships of variables and Gini
coefficient}\label{section-2-visual-relationships-of-variables-and-gini-coefficient}}

\includegraphics{FinalReport_files/figure-latex/unnamed-chunk-13-1.pdf}

\includegraphics{FinalReport_files/figure-latex/unnamed-chunk-14-1.pdf}

\includegraphics{FinalReport_files/figure-latex/unnamed-chunk-15-1.pdf}

\includegraphics{FinalReport_files/figure-latex/unnamed-chunk-16-1.pdf}

\includegraphics{FinalReport_files/figure-latex/unnamed-chunk-17-1.pdf}

\hypertarget{references}{%
\subsection*{References}\label{references}}
\addcontentsline{toc}{subsection}{References}

\hypertarget{refs}{}
\leavevmode\hypertarget{ref-autor2010}{}%
Autor, Manning, D. 2016. ``The Contribution of the Minimum Wage to U.s.
Wage Inequality over Three Decades: A Reassessment.'' \emph{American
Economic Journal: Applied Economics} 8 (1): 58--99.

\leavevmode\hypertarget{ref-budria2005}{}%
Budría, \& Pereira, S. 2005. ``Educational Qualifications and Wage
Inequality: Evidence for Europe.''

\leavevmode\hypertarget{ref-pagani2005}{}%
Dell'Aringa, \& Pagani, C. 2005. ``Regional Wage Differentials and
Collective Bargaining in Italy.'' \emph{Rivista Internazionale Di
Scienze Sociali}, 267--87.

\leavevmode\hypertarget{ref-diprete2005}{}%
DiPrete, T. A. 2008. ``Labor Markets, Inequality, and Change: A European
Perspective.'' \emph{Work and Occupations} 32 (2): 119--39.

\leavevmode\hypertarget{ref-goos2007}{}%
Goos, A., M. \& Manning. 2007. ``Lousy and Lovely Jobs: The Rising
Polarization of Work in Britain.'' \emph{Review of Economics and
Statistics} 89 (1): 118--33.

\leavevmode\hypertarget{ref-kaplan2013}{}%
Kaplan, \& Rauh, S. N. 2013. ``It's the Market: The Broad-Based Rise in
the Return to Top Talent.'' \emph{Journal of Economic Perspectives} 27
(3): 35--56.

\leavevmode\hypertarget{ref-katz1999}{}%
Katz, L. F. 1999. ``Changes in the Wage Structure and Earnings
Inequality.'' \emph{Handbook of Labor Economics} 3: 1463--1555.

\leavevmode\hypertarget{ref-laird1982random}{}%
Laird, Nan M, James H Ware, and others. 1982. ``Random-Effects Models
for Longitudinal Data.'' \emph{Biometrics} 38 (4): 963--74.

\leavevmode\hypertarget{ref-liang1986longitudinal}{}%
Liang, Kung-Yee, and Scott L Zeger. 1986. ``Longitudinal Data Analysis
Using Generalized Linear Models.'' \emph{Biometrika} 73 (1): 13--22.

\leavevmode\hypertarget{ref-rodriguez2009}{}%
Rodríguez‐Pose, \& Tselios, A. 2009. ``Education and Income Inequality
in the Regions of the European Union.'' \emph{Journal of Regional
Science} 49 (3): 411--37.

\end{document}
